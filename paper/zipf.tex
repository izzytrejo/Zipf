\documentclass[a4paper,10pt]{article}

%%% USEPACKAGES %%%



%% \usepackage{fontspec}
%% 	\setmainfont{DejaVu Sans}

\usepackage[authoryear,round,sort]{natbib}
\usepackage[german,french,russian,english]{babel}
\usepackage{textgreek}

\usepackage[margin=0.75in]{geometry}

\usepackage{graphicx}
\usepackage{wrapfig}
\usepackage{setspace}
	\onehalfspacing
\usepackage{float}

\usepackage[usenames,dvipsnames,svgnames,table,xcdraw]{xcolor}
\usepackage[font=small,labelfont=bf, labelformat=brace, labelsep=space]{caption}

\usepackage{lipsum}
\usepackage{lscape}

\usepackage{url}
\usepackage{listings}




\usepackage{hyperref}
\hypersetup{
	colorlinks,
	linkcolor={red!95!blue},
	citecolor={Cerulean!75!black},
	urlcolor={blue!80!black},
	linktoc=page
}
\usepackage[all]{nowidow}
\widowpenalty10000
\clubpenalty10000
\usepackage{multicol}

\setlength{\abovecaptionskip}{5pt plus 3pt minus 2pt} 
\setlength{\belowcaptionskip}{5pt plus 3pt minus 2pt} 


\newenvironment{boxed}[1]
{\begin{center}
		#1\\[1ex]
		\begin{tabular}{|p{0.9\textwidth}|}
			\hline\\
		}
		{ 
			\\\\\hline
		\end{tabular} 
	\end{center}
}

\newcounter{code}[section]
\newenvironment{code}[1][]{\refstepcounter{code}\par\medskip
	\noindent \textbf{OxCal~Code~\thecode. #1} \rmfamily}{\medskip}

%%% ----- OPENING -----

\title{An Analysis of Zipf's law of Frequency Distribution Across Different Languages}

%% \author{Isabella Trejo\textsuperscript{1,2,*}\\
%% 	\small{\textsuperscript{*}Corresponding author: Isabellaluztrejo@gmail.com}}\\
\author{Isabella Trejo}

\date{\normalsize{Manuscript: \textit{Radiocarbon} (\today)}}

\begin{document}

\maketitle

%%% ----- ABSTRACT -----

\begin{abstract}
	Zipf’s law was coined in 1949 by George Zipf himself. The empirical law stated that the second most common term of a text is used half as much as the most common term, and the third most common term is used a third as much as the most common term, and so on. In order to analyze Zipf’s law of frequency distribution, I had to program a word frequency distribution calculator. The question being asked here was not solely if Zipf’s law applies in a given text, but if it applies in different languages. After analyzing and processing copies of The Odyssey and The Iliad in French, Spanish, Latin, Greek, and English,I found that  
\end{abstract}

%%% ----- KEYWORDS -----

\paragraph*{Keywords:} Zipf's law, frequency distribution

%%% ----- MAIN -----
\section{Introduction}

%--- FIND LOCATION ---

Linguistic analysis is the analysis of literature and its structure. The idea is to focus on the language itself, rather than its subject matter. “The study describes the unconscious rules and processes that speakers of a language use to create spoken or written language…” It can be useful to study linguistic analysis for those who want to learn a language or translate from one language to another. “The drive behind linguistic analysis is to understand and describe the knowledge that underlies the ability to speak a given language, and to understand how the human mind processes and creates language.” Understanding the cultural and grammatical laws to speech are fundamental to a language and its’ people Knowing and analyzing language structure helps people better understand one another. \citep[e.g.,][]{Talamo2016,ConardETAL2004-Vogelherd-14C-humans,TrinkausETAL2005,DevieseETAL2017-14C-Vindija, DaviesETAL2015-trans-mosaic,StreetETAL2006-fossil-record}.  \citep[e.g.][]{FuETAL2014-Ust-Ishim,FuETAL2015-Oase}. (\autoref{fig:map}).   \citep[][p.75]{Kaminska2014-Slov-Pal-Meso}. 


\begin{figure}
	\centering
	\includegraphics[width=\textwidth]{../scripts/plot_Spanish_The_Odyssey.png}
	\caption{Map showing location of Šal’a (star) and sites with Neanderthal remains referenced in this paper: 1 -- Les Rochers-de-Villeneuve, 2 -- Ferassie (France), 3 -- Spy (Belgium), 4 --Kleine Feldhöfergrotte (Germany), 5 -- Vindija (Croatia), 6 -- Gánovce (Slovakia), 7 -- Mezmaiskaya (Russia). The site of Okladnikov is in the Altai Mountains (Russia) and not shown on the map due to its location further east.}
	\label{fig:map}
\end{figure}

%--- FIND DESCRIPTION AND PREVIOUS WORK ---

\subsection{Background}
%%% ----- BIBLIOGRAPHY -----

\bibliographystyle{apalike} 
\bibliography{BIBLIO_RJAH_Neander_Sala} 



\end{document}
